\begin{table}[H]
\centering
\caption{Protein-ligand complexes used in this study.}
\label{tab:prot-lig}
\resizebox{\textwidth}{!}{%
\begin{tabular}{rccccclccc}
\toprule
\multicolumn{1}{l}{} &
  \multicolumn{5}{c}{Protein} &
   &
  \multicolumn{3}{c}{Ligand} \\ \cline{2-6} \cline{8-10} 
PDB ID &
  Name &
  Protomers &
  $C_{\alpha}$ (\#) &
  \multicolumn{1}{l}{ENM protomers} &
  EN nodes (\#) &
   &
  Name &
  Atoms (\#) &
  Mass (amu) \\ \hline
  \href{https://www.rcsb.org/structure/4HZF}{4HZF} &
  CAP &
  \begin{tabular}[c]{@{}c@{}}A: 9-208\\ B: 10-209\end{tabular} &
  400 &
  \begin{tabular}[c]{@{}c@{}}A: 11-207\\ B: 11-207\end{tabular} &
  394 &
   &
  \href{https://www.rcsb.org/ligand/CMP}{cAMP} &
  34 &
  329.21 \\
  \href{https://www.rcsb.org/structure/1M9A}{1M9A} &
  GST &
  \begin{tabular}[c]{@{}c@{}}A: 1-216\\ B: 1-216\end{tabular} &
  432 &
  \begin{tabular}[c]{@{}c@{}}A: 4-213\\ B: 4-213\end{tabular} &
  420 &
   &
  \href{https://www.rcsb.org/ligand/GTX}{GTX} &
  56 &
  392.49 \\
  \href{https://www.rcsb.org/structure/7BQY}{7BQY} &
  M\textsuperscript{pro} &
  \begin{tabular}[c]{@{}c@{}}A: 1-301\\ B: 1-301\end{tabular} &
  602 &
  \begin{tabular}[c]{@{}c@{}}A: 2-300\\ B: 2-300\end{tabular} &
  598 &
   &
  \href{https://www.rcsb.org/ligand/PRD_002214}{N3} &
  97 &
  680.79 \\ \bottomrule
\end{tabular}%
}
\end{table}